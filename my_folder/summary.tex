%% Не менять - Do not modify
%%\input{my_folder/summary_settings} 
\chapter*[Count-me]{Реферат} % * - не нумеруем
\thispagestyle{empty}% удаляем параметры страницы
%\setcounter{sumPageFirst}{\value{page}}
%sumPageFirst \arabic{sumPageFirst}
%
%
%% Возможность проверить другие значения счетчиков - debugging
%\ref*{TotPages}~с.,
%\formbytotal{mytotalfigures}{рисун}{ок}{ка}{ков},
%\formbytotal{mytotaltables}{таблиц}{у}{ы}{},
%There are \TotalValue{mytotalfigures} figures in this document
%There are \TotalValue{mytotalfiguresInApp} figuresINAPP in this document
%There are \TotalValue{mytotaltables} tables in this document
%There are \TotalValue{mytotaltablesInApp} figuresINAPP in this document
%There are \TotalValue{myappendices} appendix chapters in this document
%\total{citenum}~библ. наименований.



%% Для того, чтобы значения счетчиков корректно отобразились, необходимо скомпилировать файл 2-3 раза
На \total{mypages}~c.,  
\formbytotal{myfigures}{рисун}{ок}{ка}{ков},
\formbytotal{mytables}{таблиц}{у}{ы}{},
\formbytotal{myappendices}{приложен}{ие}{ия}{ий}%.  

%\noindent
{\MakeUppercase{Ключевые слова: \keywordsRu}.} % Ключевые слова из renames.tex

Тема выпускной квалификационной работы: <<\thesisTitle>>.


\abstractRu % Аннотация из renames.tex

Предметом исследования является разработка системы цифровых вывесок

Объектом исследования выступает система цифровых вывесок

Цель работы: разработать систему цифровых вывесок

В процессе исследования использованы методы сравнительного анализа, моделирования и измерения.

Результатом работы является разработанная система цифровых вывесок.

Область применения результатов: Системы цифровых вывесок широко применяются в рекламной индустрии, торговых точках, а также в общественных местах, таких как аэропорты, транспортные узлы и торговые центры, для отображения рекламных сообщений, информации о продуктах и услугах, расписаниях, новостях и других информационных сообщениях.

Выводы:
\begin{itemize}
    \item разработана архитектура информационной системы
    \item проведен сравнительный анализ существующих технологий разработки требуемого ПО
    \item разработано windows приложение для показа контента
    \item разработано клиентское web приложение
    \item разработан код серверной части
    \item код серверной части развернут в облаке
    \item проведено тестирование и апробация разработанной системы
\end{itemize}
Разработанный сервис позволяет настраивать автоматическое воспроизведение медиаконтента на уделенных дисплеях под управлением windows




\printTheAbstract % не удалять


\total{mypages}~pages, 
\total{myfigures}~figures, 
\total{mytables}~tables,
\total{myappendices}~appendices%.

%\noindent
{\MakeUppercase{Keywords: \keywordsEn}.} % Ключевые слова из renames.tex 
	
The subject of the graduate qualification work is <<\thesisTitleEn>>.
	
	
\abstractEn % Аннотация из renames.tex

The subject of the research is the development of digital signage systems.

The object of the research is the digital signage system.

The aim of the work is to develop a digital signage system.

During the research, methods of comparative analysis, modeling, and measurement were used.

The result of the work is the developed digital signage system.

The application area of the results: Digital signage systems are widely used in the advertising industry, retail locations, as well as in public places such as airports, transportation hubs, and shopping centers, for displaying advertising messages, product and service information, schedules, news, and other informational messages.

Results:
\begin{itemize}
    \item The architecture of the information system has been developed.
    \item A comparative analysis of existing technologies for developing the required software has been conducted.
    \item A Windows application for content display has been developed.
    \item A client-side web application has been developed.
    \item The server-side code has been developed.
    \item The server-side code has been deployed to the cloud.
    \item Testing and validation of the developed system have been conducted.
    \item The developed service allows for the configuration of automatic playback of media content on remote displays under Windows management.
\end{itemize}

%% Не менять - Do not modify
\thispagestyle{empty}
%\setcounter{sumPageLast}{\value{page}} %сохранили номер последней страницы Задания
%\setcounter{sumPages}{\value{sumPageLast}-\value{sumPageFirst}}
%sumPageLast \arabic{sumPageLast}
%
%sumPages \arabic{sumPages}
%\restoregeometry % восстанавливаем настройки страницы
%\input{my_folder/summary_settings_restore}	% настройки - конец