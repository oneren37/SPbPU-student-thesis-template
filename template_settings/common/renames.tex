%%% Внесите свои данные - Input your data
%%
%%
\newcommand{\Author}{М.А.\,Руденко} % И.О. Фамилия автора
\newcommand{\AuthorFull}{Руденко Михаил Алексеевич} % Фамилия Имя Отчество автора
\newcommand{\AuthorFullDat}{Руденко Михаилу Алексеевичу} % Фамилия Имя Отчество автора в дательном падеже (Кому? Студенту...)
\newcommand{\AuthorFullVin}{Руденко Михаила Алексеевича} % в винительном падеже (Кого? что?  Програмиста ...)
\newcommand{\AuthorPhone}{+7-950-225-79-88} % номер телефорна автора для оперативной связи
\newcommand{\Supervisor}{А.М.\,Хахина} % И. О. Фамилия научного руководителя
\newcommand{\SupervisorFull}{Хахина Анна Михайловна} % Фамилия Имя Отчество научного руководителя
\newcommand{\SupervisorVin}{А.М.\,Хахина} % И. О. Фамилия научного руководителя  в винительном падеже (Кого? что? Руководителя ...)
% TODO уточнить должность хахиной
\newcommand{\SupervisorJob}{профессор ВШ ИСиСТ, д.т.н.} %
\newcommand{\SupervisorJobVin}{профессора ВШ ИСиСТ} % в винительном падеже (Кого? что?  Програмиста ...)
\newcommand{\SupervisorDegree}{д.т.н.} %
\newcommand{\SupervisorTitle}{} %
%%
%%
%Руководитель, утверждающий задание
\newcommand{\Head}{А.B.\,Щукин} % И. О. Фамилия руководителя подразделения (руководителя ОП)
\newcommand{\HeadDegree}{Руководитель ОП}% Только должность:
%Руководитель %ОП 
%Заведующий % кафедрой
%Директор % Высшей школы
%Зам. директора
\newcommand{\HeadDep}{} % заменить на краткую аббревиатуру подразделения или оставить пустым, если утверждает руководитель ОП

%%% Руководитель, принимающий заявление
\newcommand{\HeadAp}{А.B.\,Щукин} % И. О. Фамилия руководителя подразделения (руководителя ОП)
\newcommand{\HeadApDegree}{Руководитель ОП}% Только должность:
%Руководитель ОП 
%Заведующий кафедрой
%Директор Высшей школы
\newcommand{\HeadApDep}{} % заменить на краткую аббревиатуру подразделения или оставить пустым, если утверждает руководитель ОП
%%% Консультант по нормоконтролю
% TODO действительно ли тут должен быть Пархоменко и какая у него должность
\newcommand{\ConsultantNorm}{В.А.\,Пахмоменко} % И. О. Фамилия консультанта по нормоконтролю. ТОЛЬКО из числа ППС!
\newcommand{\ConsultantNormDegree}{} %
%%% Первый консультант
%\newcommand{\ConsultantExtraFull}{Фамилия Имя Отчетство} % Фамилия Имя Отчетство дополнительного консультанта
%\newcommand{\ConsultantExtra}{И.О.\,Фамилия} % И. О. Фамилия дополнительного консультанта
%\newcommand{\ConsultantExtraDegree}{должность, степень} %
%\newcommand{\ConsultantExtraVin}{И.О.\,Фамилию} % И. О. Фамилия дополнительного консультанта в винительном падеже (Кого? что? Руководителя ...)
%\newcommand{\ConsultantExtraDegreeVin}{должность, степень} %  в винительном падеже (Кого? что? Руководителя ...)
%%% Второй консультант
%\newcommand{\ConsultantExtraTwoFull}{Фамилия Имя Отчетство} % Фамилия Имя Отчетство дополнительного консультанта
%\newcommand{\ConsultantExtraTwo}{И.О.\,Фамилия} % И. О. Фамилия дополнительного консультанта
%\newcommand{\ConsultantExtraTwoDegree}{должность, степень} %
%\newcommand{\ConsultantExtraTwoVin}{И.О.\,Фамилию} % И. О. Фамилия дополнительного консультанта в винительном падеже (Кого? что? Руководителя ...)
%\newcommand{\ConsultantExtraTwoDegreeVin}{должность, степень} %  в винительном падеже (Кого? что? Руководителя ...)
%\newcommand{\Reviewer}{И.О.\,Фамилия} % И. О. Фамилия резензента. Обязателен только для магистров.
%\newcommand{\ReviewerDegree}{должность, степень} %
%%
%%
\renewcommand{\thesisTitle}{Проектирование и разработка системы цифровых вывесок}
\newcommand{\thesisDegree}{работа бакалавра}% дипломный проект, дипломная работа, магистерская диссертация %c 2020
\newcommand{\thesisTitleEn}{Design and development of a digital signage system} %2020
% TODO уточнить дедлайны
\newcommand{\thesisDeadline}{19.05.202X}
\newcommand{\thesisStartDate}{02.02.2024}
\newcommand{\thesisYear}{2024}
%%
%%
\newcommand{\group}{5130903/00301} % заменить вместо N номер группы
\newcommand{\thesisSpecialtyCode}{09.03.03}% код направления подготовки
\newcommand{\thesisSpecialtyTitle}{Прикладная информатика} % наименование направления/специальности
\newcommand{\thesisOPPostfix}{03} % последние цифры кода образовательной программы (после <<_>>)
\newcommand{\thesisOPTitle}{Интеллектуальные инфокоммуникационные технологии}% наименование образовательной программы
%%
%%
\newcommand{\institute}{
Институт компьютерных наук и кибербезопасности
%Институт компьютерных наук и~технологий
%Гуманитарный институт
%Инженерно-строительный институт
%Институт биомедицинских систем и технологий
%Институт металлургии, машиностроения и транспорта
%Институт передовых производственных технологий
%Институт прикладной математики и механики
%Институт физики, нанотехнологий и телекоммуникаций
%Институт физической культуры, спорта и туризма
%Институт энергетики и транспортных систем
%Институт промышленного менеджмента, экономики и торговли
}%
%%
%%




%%% Задание ключевых слов и аннотации
%%
%%
%% Ключевых слов от 3 до 5 слов или словосочетаний в именительном падеже именительном падеже множественного числа (или в единственном числе, если нет другой формы) по правилам русского языка!!!
%%
%%
\newcommand{\keywordsRu}{архитектура системы, управление контентом, wpf, react, js } % ВВЕДИТЕ ключевые слова по-русски
%%
%%
\newcommand{\keywordsEn}{system architecture, content management, wpf, react, js } % ВВЕДИТЕ ключевые слова по-английски
%%
%%
%% Реферат ОТ 1000 ДО 1500 знаков на русский или английский текст
%%
%Реферат должен содержать:
%- предмет, тему, цель ВКР;
%- метод или методологию проведения ВКР:
%- результаты ВКР:
%- область применения результатов ВКР;
%- выводы.

\newcommand{\abstractRu}{В данной работе изложена сущность подхода к созданию динамического информационного портала на основе использования открытых технологий Apache, MySQL и PHP. Даны общие понятия и классификация IT-систем такого класса. Проведен анализ систем-прототипов. Изучена технология создания указанного класса информационных систем. Разработана конкретная программная реализация динамического информационного портала на примере портала выбранной тематики...} % ВВЕДИТЕ текст аннотации по-русски
%%
%%
\newcommand{\abstractEn}{In the given work the essence of the approach to creation of a dynamic information portal on the basis of use of open technologies Apache, MySQL and PHP is stated. The general concepts and classification of IT-systems of such class are given. The analysis of systems-prototypes is lead. The technology of creation of the specified class of information systems is investigated. Concrete program realization of a dynamic information portal on an example of a portal of the chosen subjects is developed...} % ВВЕДИТЕ текст аннотации по-английски


%%% РАЗДЕЛ ДЛЯ ОФОРМЛЕНИЯ ПРАКТИКИ
%Место прохождения практики
\newcommand{\PracticeType}{Отчет о прохождении % 
	%стационарной производственной (технологической (проектно-технологической)) %
	такой-то % тип и вид ЗАМЕНИТЬ
	практики}

\newcommand{\Workplace}{СПбПУ, ИКНТ, ВШИСиСТ} % TODO Rename this variable

% Даты начала/окончания
\newcommand{\PracticeStartDate}{%
дд.мм.гггг%
%	22.06.2020
}%
\newcommand{\PracticeEndDate}{%
	дд.мм.гггг%
%	18.07.2020%
}%
%%

\newcommand{\School}{
	Название высшей школы
%	Высшая школа интеллектуальных систем и~суперкомпьютерных~технологий 
}
\newcommand{\practiceTitle}{Тема практики}


%% ВНИМАНИЕ! Необходимо либо заменить текст аннотации (ключевых слов) на русском и английском, либо удалить там весь текст, иначе в свойства pdf-отчета по практике пойдет шаблонный текст.

%%% Не меняем дальнейшую часть - Do not modify the rest part
%%
%%
%%
%%
\ifnumequal{\value{docType}}{1}{% Если ВКР, то...
	\newcommand{\DocType}{Выпускная квалификационная работа}
	\newcommand{\pdfDocType}{\DocType~(\thesisDegree)} %задаём метаданные pdf файла
	\newcommand{\pdfTitle}{\thesisTitle}
}{% Иначе 
	\newcommand{\DocType}{\PracticeType}
	\newcommand{\pdfDocType}{\DocType} %задаём метаданные pdf файла
	\newcommand{\pdfTitle}{\practiceTitle}
}%
\newcommand{\HeadTitle}{\HeadDegree~\HeadDep}
\newcommand{\HeadApTitle}{\HeadApDegree~\HeadApDep}
\newcommand{\thesisOPCode}{\thesisSpecialtyCode\_\thesisOPPostfix}% код образовательной программы
\newcommand{\thesisSpecialtyCodeAndTitle}{\thesisSpecialtyCode~\thesisSpecialtyTitle}% Код и наименование направления/специальности
\newcommand{\thesisOPCodeAndTitle}{\thesisOPCode~\thesisOPTitle} % код и наименование образовательной программы
%%
%%
\hypersetup{%часть болка hypesetup в style
		pdftitle={\pdfTitle},    % Заголовок pdf-файла
		pdfauthor={\AuthorFull},    % Автор
		pdfsubject={\pdfDocType. Шифр и наименование направления подготовки: \thesisSpecialtyCodeAndTitle. \abstractRu},      % Тема
		pdfcreator={LaTeX, SPbPU-student-thesis-template},     % Приложение-создатель
%		pdfproducer={},  % Производитель, Производитель PDF % будет выставлена автоматически
		pdfkeywords={\keywordsRu}
}
%%
%%
%% вспомогательные команды
\newcommand{\firef}[1]{рис.\ref{#1}} %figure reference
\newcommand{\taref}[1]{табл.\ref{#1}}	%table reference
%%
%%
%% Архивный вариант задания ключевых слов, аннотации и благодарностей 
% Too hard to export data from the environment to pdf-info
% https://tex.stackexchange.com/questions/184503/collecting-contents-of-environment-and-store-them-for-later-retrieval
%заменить NewEnviron на newenvironment для распознавания команды в TexStudio
%\NewEnviron{keywordsRu}{\noindent\MakeUppercase{\BODY}}
%\NewEnviron{keywordsEn}{\noindent\MakeUppercase{\BODY}}
%\newenvironment{abstractRu}{}{}
%\newenvironment{abstractEn}{}{}
%\newenvironment{acknowledgementsRu}{\par{\normalfont \acknowledgements.}}{}
%\newenvironment{acknowledgementsEn}{\par{\normalfont \acknowledgementsENG.}}{}


%%% Переопределение именований %%% Не меняем - Do not modify
%\newcommand{\Ministry}{Минобрнауки России} 
\newcommand{\Ministry}{Министерство науки и высшего образования Российской~Федерации} %с 2020
\newcommand{\SPbPU}{Санкт-Петербургский политехнический университет Петра~Великого}
\newcommand{\SPbPUOfficialPrefix}{Федеральное государственное автономное образовательное учреждение высшего образования}
\newcommand{\SPbPUOfficialShort}{ФГАОУ~ВО~<<СПбПУ>>}
%% Пробел между И. О. не допускается.
\renewcommand{\alsoname}{см. также}
\renewcommand{\seename}{см.}
\renewcommand{\headtoname}{вх.}
\renewcommand{\ccname}{исх.}
\renewcommand{\enclname}{вкл.}
\renewcommand{\pagename}{Pages}
\renewcommand{\partname}{Часть}
\renewcommand{\abstractname}{\textbf{Аннотация}}
\newcommand{\abstractnameENG}{\textbf{Annotation}}
\newcommand{\keywords}{\textbf{Ключевые слова}}
\newcommand{\keywordsENG}{\textbf{Keywords}}
\newcommand{\acknowledgements}{\textbf{Благодарности}}
\newcommand{\acknowledgementsENG}{\textbf{Acknowledgements}}
\renewcommand{\contentsname}{Content} % 
%\renewcommand{\contentsname}{Содержание} % (ГОСТ Р 7.0.11-2011, 4)
%\renewcommand{\contentsname}{Оглавление} % (ГОСТ Р 7.0.11-2011, 4)
\renewcommand{\figurename}{Рис.} % Стиль СПбПУ
%\renewcommand{\figurename}{Рисунок} % (ГОСТ Р 7.0.11-2011, 5.3.9)
\renewcommand{\tablename}{Таблица} % (ГОСТ Р 7.0.11-2011, 5.3.10)
%\renewcommand{\indexname}{Предметный указатель}
\renewcommand{\listfigurename}{Список рисунков}
\renewcommand{\listtablename}{Список таблиц}
\renewcommand{\refname}{\fullbibtitle}
\renewcommand{\bibname}{\fullbibtitle}

\newcommand{\chapterEnTitle}{Сhapter title} % <- input the English title here (only once!) 
\newcommand{\chapterRuTitle}{Название главы}          % <- введите 
\newcommand{\sectionEnTitle}{Section title} %<- input subparagraph title in english
\newcommand{\sectionRuTitle}{Название подраздела} % <- введите название подраздела по-русски
\newcommand{\subsectionEnTitle}{Subsection title} % - input subsection title in english
\newcommand{\subsectionRuTitle}{Название параграфа} % <- введите название параграфа по-русски
\newcommand{\subsubsectionEnTitle}{Subsubsection title} % <- input subparagraph title in english
\newcommand{\subsubsectionRuTitle}{Название подпараграфа} % <- введите название подпараграфа по-русски